\section{Arquitectura de Comportamientos}

\subsection{Introducci\'on}
Como se dijo en las secciones anteriores, el objetivo de \'este trabajo
es desarrollar un robot aut\'onomo y reactivo que recolecte basura en un entorno
estructurado pero din\'amico, debido a que la arena ( lugar donde se mueve el robot )
es transitado por personas y es al aire libre.

En \'esta secci\'on de \emph{Arquitecturas de Comportamientos} se van
a detallar las acciones que deber\'a hacer el robot y c\'omo organizar las mismas
para cumplir la meta de recolectar basura y ser aut\'onomo, es decir, decidir por
s\'i mismo las acciones a realizar en cada momento y ser capaz de mantenerse cargado
para poder continuar recolectando.

En la secci\'on \ref{inv_prev} se analizar\'an
papers relacionados con este desarrollo, desde la forma de organizar los comportamientos,
la definici\'on y composici\'on de los mismos, hasta su implementaci\'on. En la secci\'on
\ref{arq_prop} se detallar\'a la arquitectura elegida para llevar a cabo y organizar los
comportamientos elegidos, y las ventajas y desventajas de usar la misma. La secci\'on
\ref{comportamientos} detalla uno por uno los comportamientos elegidos que tendr\'a el
robot, y sus correspondientes implementaciones en pseudoc\'odigo. En la secci\'on
\ref{odometry} se explicar\'a que es la odometr\'ia, para qu\'e sirve y qu\'e ventajas y desventajas
tiene usarla. Tambi\'en se detallar\'a el test utilizado para mejorar la eficiencia de la misma.
La secci\'on \ref{interfaces} se describir\'a c\'omo se estructur\'o el controlador del robot
para que se pueda ir desarrollando y probando el mismo con un simulador y minimizar el trabajo
al pasarlo a el robot f\'isico. Los resultados obtenidos de performance y eficiencia del controlador
desarrollado se obtuvieron de la simulaci\'on, y se pueden ver en la secci\'on \ref{results}. Finalmente,
en la secci\'on \ref{comp_conclusion} se sacan conclusiones de los resultados obtenidos y las dificultades
encontradas a lo largo del desarrollo de \'este trabajo pr\'actico final.

\newpage

\subsection{Investigaciones previas}
\label{inv_prev}

A continuaci\'on se van a presentar trabajos realizados por otros autores relacionados en alguna
forma con el trabajo que nosotros realizamos. Primero se da una breve descripci\'on del trabajo
del otro autor y luego se comparan con nuestro trabajo, viendo similitudes y diferencias entre ambos.

\subsubsection{Desarrollo de comportamientos no triviales en robots reales : Robot recolector de basura - Stefano Nolfi \cite{nolfi:evolving}}
En \'este paper se muestra el uso de un Khepera con el m\'odulo Gripper en una arena delimitada por
paredes para la recolecci\'on de "basura". Dicho robot se puede ver en la figura \ref{fig:khepera}.
\begin{figure}[htp]
\begin{center}
\includegraphics[scale=0.6]{comportamientos/gripperAH.png}
\caption{Robot Khepera con el m\'odulo Gripper}
\label{fig:khepera}
\end{center}
\end{figure}
La base tiene una fuente de luz asociada y el objetivo del
robot es llevar peque\~nos objetos a un lugar de dep\'osito. Se utiliza \emph{aprendizaje por refuerzo} para
asociar las velocidades angulares a los diferentes tama\~nos de las "basuras". Para obtener el comportamiento
deseado se hace uso de \emph{algoritmos gen\'eticos} y \emph{redes neuronales}. Una vez obtenido el mismo mediante
simulaci\'on en Webots, se lo prob\'o en el Khepera f\'isico.
Comparando el trabajo de Stefano Nolfi con el nuestro podemos observar que:
\begin{itemize}

\item{Existencia de un dep\'osito:} Ambos tienen un dep\'osito en el cual dejar la basura recogida y una forma
de identificarla: una fuente de luz usada por Nolfi y en nuestro trabajo, al llegar al final de una determinada
l\'inea de la arena.

\item{Autonom\'ia:} Ambos son aut\'onomos en el sentido que no son manejados por un ente externo. En cuanto
a la recarga de la bater\'ia, Nolfi no explicita si es asistida o no; en nuestro caso, el robot se encarga
de ir a la base de recarga una vez detectada la falta de energ\'ia.

\item{M\'etodo de Recolecci\'on:} En el trabajo de Nolfi se utiliza un m\'odulo agregado al Khepera, que simula
un brazo con dos dedos, la recolecci\'on se realiza juntando ambos y la liberaci\'on se realiza separando ambos.
Nuestro trabajo se basa m\'as en la actividad humana que en el comportamiento humano para realizar este comportamiento,
ya que podr\'ia compararse con un recolector de basura que primero guarda lo encontrado usando una pala en su tacho
y luego descarga su tacho en un dep\'osito de mayor tama\~no.

\item{M\'etodo de aprendizaje:} Nuestro trabajo no utiliza alguna forma de aprendizaje por refuerzo o algoritmo
evolutivo. No es el caso de Nolfi, que utiliza ambas t\'ecnicas para evolucionar el comportamiento deseado.

\item{Robot Utilizado:} Como ya se dijo, Nolfi utiliz\'o un Khepera en la simulaci\'on. En nuestro trabajo
utilizamos un E-puck para la simulaci\'on, una versi\'on nueva del Khepera, pero sin el m\'odulo Gripper.
\begin{figure}[htp]
\begin{center}
\includegraphics[scale=0.6]{comportamientos/e-puck.png}
\caption{Robot E-puck}
\label{fig:epuck}
\end{center}
\end{figure}

\end{itemize}

\subsubsection{Arquitectura de control para un Robot Aut\'onomo M\'ovil - Neves And Oliveira \cite{Neves97acontrol}}
Neves y Oliveira describen en su trabajo una arquitectura de control basada en comportamientos para un
robot m\'ovil en un ambiente din\'amico, utilizando muchos aspectos de la arquitectura \emph{Subsumption} propuesta
por \emph{Brooks} y que fue usada por nosotros al realizar \'este trabajo explicada en \ref{arq_prop}.
La arquitectura propuesta, o \emph{Control System Arquitecture}, se basa tambi\'en en la teor\'ia de
\emph{The Society of Mind}, escrita por \emph{Minsky}, donde el sistema es visto como una sociedad de agentes,
cada uno con una particular competencia y que colaboran entre ellos para ayudar a la sociedad a alcanzar
su meta.
La arquitectura est\'a compuesta por tres niveles: un nivel reflexivo, uno reactivo, y otro cognitivo, aumentando
la complejidad al igual que el orden en que fueron presentados.
\\
El nivel reflexivo incluye aquellos comportamientos
innatos, es decir, act\'uan directamente como est\'imulo-respuesta. El segundo nivel, el reactivo, est\'a
compuesto por agentes que responden r\'apidamente a los est\'imulos ya que requieren poco nivel de procesamiento.
Finalmente, en el nivel cognitivo, se encuentran los agentes encargados de guiar y administrar los comportamientos
reactivos de forma tal que el robot muestre un comportamiento orientado.
\\
Aunque la arquitectura explicada es similar a la utilizada en nuestro trabajo, \'esta organizaci\'on de los
comportamientos en capas no la utilizamos, aunque podemos divisar que algunos de los comportamientos que propusimos
corresponden a la primera capa de reflexi\'on, como por ejemplo el evitamiento de
obst\'aculos; otros podr\'ian incluirse en la segunda capa de comportamientos reactivos, como lo es deambular y
finalmente en la \'ultima capa estar\'ia el reconocimiento de objectos debido a su gran demanda de procesamiento.

\subsubsection{Path Planning usando Algoritmos Gen\'eticos - Salvatore Candido \cite{salvatore}}
En este paper se describe la utilizaci\'on de algor\'itmos gen\'eticos para resolver el problema de Path Planning.
\'Este problema consiste en armar un plan, una secuencia de acciones de forma tal que a partir de un punto de origen y siguiendo
ese plan, se llegue a un segundo punto, el de destino. Debido a la componente din\'amica de nuestro trabajo, siempre
tratamos de mantener los comportamientos del robot lo m\'as reactivos posibles, por lo que armar un plan no 
ser\'ia el mejor approach para ser utilizado. Por el contrario, el uso de algor\'itmos gen\'eticos puede ser
una herramienta que podr\'ia llegar a ser utilizada en una futura continuaci\'on de nuestro trabajo y que no
utilizamos por el tiempo que estimamos que nos hubiese demandado.

\subsubsection{Navegaci\'on predictiva de un robot aut\'onomo - Foka And Trahanias \cite{Foka02predictiveautonomous}}
La navegaci\'on predictiva nace como una posible soluci\'on al problema de un robot navegando en un ambiente
con muchas personas y obst\'aculos, tal como lo es el ambiente en el cual navegar\'a el robot resultante
de nuestro trabajo.
\\
La forma en que es implementada por los autores es mediante un \emph{POMDP}, es decir, 
un proceso de decisi\'on de markov parcialmente observable. Cabe destacar estas dos \'ultimas palabras, ya
que indican la naturaleza del ambiente: hay incertidumbre o falta de informaci\'on acerca de ciertas variables
del entorno. En este caso, se utiliza el \emph{POMDP} para manejar tanto la navegaci\'on del robot como el
evitamiento de obst\'aculos, un punto en que se diferencia de lo propuesto por nosotros, que es tratar el
\emph{wandering} de forma separada del m\'odulo de evitamiento de obst\'aculos. \'Esto, en parte, se debi\'o
a la arquitectura utilizada y que desde ese punto de vista, ambos comportamientos son muy lejanos.

\subsubsection{Algoritmo de navegaci\'on y evitamiento de obst\'aculos en un entorno desconocido - Clark Et. al \cite{clark}}
En este paper se presentan dos algoritmos complementarios para la navegaci\'on en ese tipo de entornos.
\\
El primero consiste en la navegaci\'on y un mapeo del entorno que garantiza una cobertura completa de una arena
cuyas ubicaciones de la paredes no se conocen \emph{a priori}. Consiste b\'asicamente en un seguimiento de las
paredes complementado por una variaci\'on de \emph{flood filling} para asesgurarse la cobertura completa de la
arena. El algoritmo completo se puede apreciar en la figura \ref{fig:clark}
\begin{figure}[htp]
\begin{center}
\includegraphics[scale=0.8]{comportamientos/clarkDiagram.png}
\caption{Algoritmo completo}
\label{fig:clark}
\end{center}
\end{figure}
\\
Un aut\'omata con aprendizaje estoc\'astico es el algoritmo que complementa al primero, y su objetivo es
el evitamiento de obst\'aculos. Para \'esto, se utiliza un mecanismo de recompensa/castigo de forma tal
que se adapten las probabilidades de las acciones a tomar. 
\\
En relaci\'on con nuestro trabajo, hay ciertas similitudes y diferencias, enumeradas a continuaci\'on:
\begin{itemize}
\item{}Ambos presentan un mecanismo de "seguimiento de", en nuestro caso lo hacemos con l\'ineas y basuras, en
el paper se utiliza con paredes. Sin embargo, se utilizan para objetivos diferentes. Nosotros lo usamos para
dirigirnos a la base ya que tratamos de mantener el conociemiento que el robot tiene sobre el mundo lo m\'as
acotado posible, o, en el caso de las basuras, para recolectarlas. En el paper se utiliza el mecanismo de seguir
las l\'ineas para obtener un modelo del mundo, algo que puede llegar a servir mucho en ambientes no tan din\'amicos
como el nuestro, raz\'on por la cual no elegimos implementarlo.
\item{}Tambi\'en se coincide en la existencia de un m\'etodo para evitar obst\'aculos. En el paper se implementa como
un aut\'omata que va aprendiendo seg\'un los premios o castigos que recibe; en nuestro caso, el comportamiento es puramente
reactivo y reacciona en base a los valores de los sensores de distancia, pero no tiene memoria.
\end{itemize}

\newpage
\subsection{Arquitectura propuesta}
\label{arq_prop}
Una arquitectura basada en comportamientos define la forma en que los mismos son especificados, desde
su granularidad ( que tan complejo o simple es un comportamiento ), la base para su especificaci\'on,
el tipo de respuesta y la forma en que se coordinan.
\\
La arquitectura que elegimos para desarrollar nuestro trabajo es \emph{Subsumption}, desarrollada por
Rodney Brooks a mediados de 1980. \'Esta arquitectura est\'a basada en comportamientos puramente reactivos,
rompiendo as\'i con el esquema que estaba de moda en la \'epoca de \emph{sensar-planear-actuar}. Algunos
de los principios propuestos que fueron tenidos en cuenta durante el desarrollo de nuestro trabajo son:
\begin{itemize}
\item{}Un comportamiento complejo no es necesariamente el producto de un complejo sistema de control
\item{}El mundo es el mejor modelo del mismo
\item{}La simplicidad es una virtud
\item{}Los sistemas deben ser construidos incrementalmente
\end{itemize}
Cada comportamiento es un par est\'imulo-respuesta. Tal como se observa en la Figura \ref{fig:behaviour},
cada est\'imulo o respuesta puede ser inhibido o suprimida por otros comportamientos activos.
Adem\'as, cada comportamiento recibe una se\~nal de reset, que vuelve al comportamiento que
recibi\'o esta se\~nal a su estado original.
\begin{figure}[htp]
\begin{center}
\includegraphics[scale=0.85]{comportamientos/behaviour.png}
\caption{Esquema de comportamiento}
\label{fig:behaviour}
\end{center}
\end{figure}
El nombre \emph{Subsumption} proviene de la forma en que los comportamientos son coordinados entre s\'i. Hay
una jerarqu\'ia donde los comportamientos de la arquitectura tienen mayor o menor prioridad seg\'un su posici\'on.
Los comportamientos de los menores niveles no tienen conocimiento de los comportamientos de las capas de mayor
nivel. Gracias a esto, se puede plantear un dise\~no incremental, brindando flexibilidad, adaptaci\'on y paralelismo
al desarrollo e implementaci\'on de los comportamientos.
\\
La idea a seguir es que el mundo sea el principal medio de comunicaci\'on entre los comportamientos, dado que la respuesta
de un comportamiento ante est\'imulo resulta en un cambio en el mundo y, por lo tanto, en la relaci\'on del robot con el mismo,
ya que el robot en su pr\'oximo paso sensar\'a otro estado del mundo.
\\
El procedimiento b\'asico para dise\~nar y desarrollar comportamientos para robots con esta arquitectura es sencillo:
\begin{enumerate}
\item Especificar cualitativamente la forma en que el mundo responde al mundo, es decir, el comportamiento que realizar\'a.
\item Descomponer la especificaci\'on como un conjunto de acciones disjuntas.
\item Determinar la granularidad del comportamiento, analizando en que nivel de la jerarqu\'ia existente se encontrar\'a y 
cu\'antas acciones disjuntas es necesario llevar a cabo para el cumplimiento de la tarea.
\end{enumerate}

Un ejemplo de \'esta arquitectura se puede observar en \ref{fig:subsumptionExample}. En la misma hay 4 comportamientos:
Homing, Pickup, Avoiding y Wandering. Las l\'ineas que entran a cada comportamiento son los est\'imulos ante los cuales
se activan y las salidas de los mismos son las se\~nales de respuesta correspondientes. La se\~nal de respuesta de la arquitectura
es la l\'inea que sale por la derecha de la caja (Arquitectura) que contiene la relaci\'on entre los comportamientos.
Puede verse como Homing inhibe la salida de Pickup ya que su salida entra al supresor (denotado con un c\'irculo con una S dentro)
de la salida de Pickup y por lo tanto, las salidas de los dem\'as comportamientos, ya que su prioridad es mayor al resto. En el
caso que no est\'en presentes los est\'imulos de Homing, Pickup y Avoiding, no hay inhibici\'on en la salida de Wandering, y por lo
tanto se lleva a cabo el comportamiento de Wandering.
\begin{figure}[htp]
\begin{center}
\includegraphics[scale=1.0]{comportamientos/subsumptionExample.png}
\caption{Un ejemplo de Subsumption}
\label{fig:subsumptionExample}
\end{center}
\end{figure}


\newpage

\subsection{Comportamientos e implementaciones}
\label{comportamientos}

Una vez que decidimos utilizar la arquitectura explicada en la secci\'on \ref{arq_prop} para nuestro proyecto, tuvimos que analizar:
\begin{itemize}
\item{}Forma de implementaci\'on de la arquitectura en c\'odigo
\item{}Comportamientos a realizar
\item{}Ordenes de inhibici\'on y supresi\'on entre los mismos
\item{}Forma de implementaci\'on de los mismos
\item{}Orden de implementaci\'on
\end{itemize}

%Detallar que se hizo con cada item
Para implementar la arquitectura decidimos asignarle un $ID$ num\'erico diferente a cada comportamiento.
Tambi\'en tomamos la decisi\'on de elegir como comportamiento activo en el instante $t$,
aquel comportamiento que est\'e activo en ese instante y tenga mayor $ID$,
suprimiendo as\'i el resto de los comportamientos ( con un $ID$ menor ) que podr\'ian estar activos.

%Explicar como se relaciona el requerimiento (robot recolector de basura autonomo) con los comportamientos que pusimos
El requerimiento de este proyecto es la realizaci\'on de un robot aut\'onomo que recolecte
basura de su entorno din\'amico pero estructuralmente fijo. De aqu\'i se infieren algunos de los
comportamientos que debe tener el robot:
\begin{itemize}
	\item{\emph{Recolectar basura}}
	\item{\emph{Recargar bater\'ia}:} Por ser aut\'onomo, debe poder ser capaz de recargarse s\'olo
								para poder continuar con su actividad.
	\item{\emph{Wandering}:} El robot no es controlado por control, ya que es aut\'onomo, por lo que
						debe poder recorrer el entorno por s\'i mismo.
	\item{\emph{Evitamiento de obst\'aculos}:} Debido a la naturaleza del entorno, el robot debe
						ser capaz de navegar sin chocarse contra los l\'imites del entorno
						ni con las personas que circulan por el mismo.
\end{itemize}

El comportamiento de recolectar basura y el requerimiento de la autonom\'ia llevan a su vez
a la aparici\'on de m\'as comportamientos: \emph{Descargar basura} y \emph{Ir hacia basura}.
\\
La figura \ref{fig:architecture} muestra los comportamientos implementados y su orden de jerarqu\'ia.
Se puede ver que hay m\'as comportamientos de los detallados anteriormente. \'Esto se debi\'o a que la forma
en que se implementamos los comportamientos b\'asicos del robot nos requiri\'o el desarrollo
de comportamientos auxiliares.
\\
\begin{figure}[htp]
\begin{center}
\includegraphics[scale=0.5]{comportamientos/behavioursArchitecture.png}
\caption{Arquitectura de Comportamientos}
\label{fig:architecture}
\end{center}
\end{figure}
\\
Primero implementamos \emph{Wandering} debido a, en una primera aproximaci\'on, la sencillez
del mismo. Luego implementamos \emph{Evitamiento de obst\'aculos} para lograr que el robot pueda
navegar sin problemas por el arena. Como el comportamiento de recolectar e ir hacia la basura
dependend\'ia del m\'odulo de reconocimiento de objetos y el mismo estaba siendo desarrollado
en paralelo, se decidi\'o implementar el comportamiento de \emph{Recargar bater\'ia} y \emph{Descargar basura}.
Una vez que tuvimos la primera implementaci\'on funcional del reconomiento de objetos, procedimos a
desarrollar \emph{Ir hacia basura} y \emph{Recolectar Basura}.
\\
A continuaci\'on detallamos los comportamientos indicados en la figura \ref{fig:architecture},
as\'i como la implementaci\'on en \textit{pseudo-codigo} de los mismos y detalles tenidos
en cuenta para la realizaci\'on de los mismos.
%A continuacion vamos a explicar cada comportamiento.....

\subsubsection{Wandering}
\label{wandering}

\paragraph{Detalle del comportamiento}
En una primer aproximaci\'on de \emph{Wandering}, s\'olo nos preocupamos por ir hacia adelante
ya que eventualmente, el robot encuentra un obst\'aculo y realiza un giro cambiando la direcci\'on
del robot.
\\
Los resultados de la simulaci\'on nos indicaron que el robot no recorr\'ia ciertas zonas o las recorr\'ia
despu\'es de un largo tiempo, lo que nos llev\'o a un segundo approach. El mismo tiene en cuenta el hecho
de que el robot posee una c\'amara y por lo tanto se puede llevar un seguimiento de los lugares que m\'as
recientemente visit\'o o las zonas que hace mucho tiempo no visita.
\\
\'Este approach, en cierta forma, genera un modelo del mundo, un hecho que conflict\'ua con uno de
los principios propuestos a seguir en la secci\'on \ref{arq_prop}. Para minimizar el conflicto, decidimos
mantener al m\'inimo la informaci\'on almacenada para el funcionamiento del algoritmo, es decir, por cada
zona de arena s\'olo mantenemos el timestamp de la \'ultima vez que el robot la visit\'o.

\paragraph{Implementaci\'on del comportamiento}
La implementaci\'on del segundo approach, en \emph{pseudo-c\'odigo} es la siguiente:
\begin{verbatim}
por cada paso
    zona = pedir_zona_vista(camara)
    marcar_zona_como_vista(modelodelmundo,zona)
    ultimazonavisitada = pedir_ultima_zona_visitada(modelodelmundo)
    velocidades = calcular_velocidades_de_ruedas(ultimazonavisitada)
    poner_velocidades_en_ruedas(velocidades)
\end{verbatim}


\subsubsection{Enfocar Basura}
\label{focus_garbage}
\paragraph{Detalle del comportamiento}
\paragraph{Implementaci\'on del comportamiento}

\subsubsection{Ir a Basura}
\label{go_to_garbage}
\paragraph{Detalle del comportamiento}
\paragraph{Implementaci\'on del comportamiento}

\subsubsection{Recolectar Basura}
\label{collect_garbage}
\paragraph{Detalle del comportamiento}
\paragraph{Implementaci\'on del comportamiento}

\subsubsection{Ir a zona de descarga de basura}
\label{go_to_unload_zone}

\paragraph{Buscar l\'inea}
\label{find_line}
\subparagraph{Detalle del comportamiento}
\subparagraph{Implementaci\'on del comportamiento}

\paragraph{Entrar a l\'inea}
\label{enter_line}
\subparagraph{Detalle del comportamiento}
\subparagraph{Implementaci\'on del comportamiento}

\paragraph{Seguir l\'inea}
\label{follow_line}
\subparagraph{Detalle del comportamiento}
\subparagraph{Implementaci\'on del comportamiento}

\subsubsection{Descargar Basura}
\label{unload_garbage}
\paragraph{Detalle del comportamiento}
\paragraph{Implementaci\'on del comportamiento}

\subsubsection{Ir a base de recarga de bater\'ia}
\label{go_to_recharge}
\paragraph{Detalle del comportamiento}
\paragraph{Implementaci\'on del comportamiento}

\subsubsection{Cargar Bater\'ia}
\label{recharge_battery}
\paragraph{Detalle del comportamiento}
\paragraph{Implementaci\'on del comportamiento}

\subsubsection{Recalibrarse}
\label{recalibrate}
\paragraph{Detalle del comportamiento}
\paragraph{Implementaci\'on del comportamiento}

\subsubsection{Evitar Obst\'aculos}
\label{avoid_obstacles}
\paragraph{Detalle del comportamiento}
\paragraph{Implementaci\'on del comportamiento}

\subsubsection{Salir de situaciones no deseadas}
\label{out_of_unwanted_situations}
\paragraph{Detalle del comportamiento}
\paragraph{Implementaci\'on del comportamiento}



\newpage
\subsection{Odometr\'ia}
\label{odometry}
Para saber donde se encuentra el robot en cierto momento, se utiliza odometr\'ia.
\\
\'Esta se basa en la medici\'on de las cuentas dadas por un motor para obtener el desplazamiento realizado 
por la rueda asociada al mismo. Dado que el motor que usamos fue un motor diferencial para cada rueda,
\'esto es posible de realizar.
\\
A diferencia de los m\'etodos de posicionamiento absoluto, la odometr\'ia
da una estimaci\'on del desplazamiento \emph{local} a la ubicaci\'on anterior del robot, por lo cual \emph{un error}
en una estimaci\'on \emph{se propaga} hacia las siguientes estimaciones. 
\\
Para calcular la posici\'on en el instante n $P_n$ y la orientaci\'on $O_n$ en base a la posici\'on en el 
instante (n-1) $P_{n-1}$ y la correspondiente orientaci\'on $O_{n-1}$, usamos las siguientes f\'ormulas:
\begin{eqnarray}
d_l = \frac{(e_l(n) - e_l(n-1)) * R_l}{EncRes} \\
d_r = \frac{(e_r(n) - e_r(n-1)) * R_r}{EncRes} \\
lc = \frac{d_r + d_l}{2} \\
P_n = P_{n-1} + (lc * \cos{O_{n-1}},lc * \sin{O_{n-1}}) \\
O_n = O_{n-1} + \frac{d_r - d_l}{dbw}
\end{eqnarray}
donde $d_l$ y $d_r$ son las distancias recorridas por las ruedas izquierda y derecha respectivamente. Ambas
son calculadas teniendo en cuenta los valores anteriores y actuales de los encoders $e_i(n)$, el radio de la rueda $R_i$ y
la resoluci\'on del encoder $EncRes$. $dbw$ es la distancia entre ruedas.
\\
Los errores que influyen en el c\'alculo de la odometr\'ia pueden ser de dos tipos:

\begin{itemize}

\item{Sistem\'aticos:} Son aquellos que pueden ser corregidos o tenidos en cuenta para disminuir el error.

\item{No sistem\'aticos:} Son aquellos que pueden intentarse corregir pero \emph{no} eliminar.

\end{itemize}

Decidimos tener en cuenta dos errores sistem\'aticos para disminuir el error en la odometr\'ia :
\begin{itemize}
\item{Incerteza sobre la distancia entre las ruedas ($dbw$)}
\item{Ruedas con radios diferentes ($R_l$ y $R_r$)}
\end{itemize}
Tuvimos en cuenta estos errores dado que son los que m\'as contribuyen al error acumulado a lo
largo del trayecto del robot.
Para corregir \'estas fuentes de error, utilizamos el \emph{test del camino
bidireccional describiendo un cuadrado} (UMBmark)\footnote{http://www-personal.umich.edu/~johannb/Papers/umbmark.pdf}.
Dado que en el momento de realizar el test no hab\'ia un robot f\'isico, decidimos realizar el test sobre
un e-puck simulado en Webots.
As\'i fuimos obteniendo los valores de correcci\'on para los di\'ametros de las ruedas $c_i$ y la correci\'on
sobre la distancia entre las ruedas $c_{dbw}$ hasta que el error sistem\'atico m\'aximo dejara de disminuir. En la
figura \ref{fig:errsist} se puede ver c\'omo disminuye el error a lo largo de las iteraciones.

\begin{figure}[htp]
\begin{center}
%\includegraphics[scale=0.6]{comportamientos/errsist.eps}
\includegraphics[scale=0.3]{comportamientos/unk.jpg}
\caption{Error Sistem\'atico M\'aximo a lo largo de las iteraciones}
\label{fig:errsist}
\end{center}
\end{figure}

En la \'ultima iteraci\'on obtuvimos los coeficientes de correcci\'on $c_l = 0.99998703$, $c_r = 1.00001297$
y $c_{dbw} = 1.092171094$ para ruedas con radio $R_l = R_r = 0.205$, una distancia entre ruedas de $0.052$ y
una resoluci\'on de encoder $EncRes = 159.23$.

\subsubsection{Problemas con la odometr\'ia}
\label{odometry:problems}
A LA LA LALALA

\newpage
\subsection{Interfaces con hardware y m\'odulo de reconocimiento de objetos}
\label{interfaces}
Decidimos hacer que el controlador encargado de realizar los comportamientos sea independiente
de la forma con la que se implemente el hardware y la forma con que se implemente el reconocimiento
de los objetos. Para \'esto definimos interfaces, de forma tal que el controlador las mismas
y tanto el hardware como el m\'odulo de reconocimiento de objetos puedan cambiar su implementaci\'on
pero brindando siempre la informaci\'on que necesita el controlador para poder realizar
los comportamientos. \'Esta decisi\'on nos posibilit\'o realizar un controlador que sea capaz de ser
ejecutado tanto en Webots, donde se hizo el desarroll\'o, como en la base real del robot. Cabe aclarar
que el traspaso de la simulaci\'on a la realidad no es instant\'anea, ya que hay que calibrar los dispositivos,
realizar nuevamente la odometr\'ia, entre otras cosas, pero el trabajo demandado es mucho menor ya
que una correcci\'on en la l\'ogica de los comportamientos se puede observar tanto en un simulador como
en la realidad.

\subsubsection{Interfaz con hardware}
La interfaz con el hardware se basa en tener clases encargadas de obtener los valores de los sensores
o del accionar de los actuadores.
\\En la implementaci\'on de la interfaz que corrimos en Webots, hic\'imos
llamadas al controlador del simulador, que utiliza sensores y actuadores simulados.
\\En la implementaci\'on que se comunica con el robot f\'isico,
las llamadas las hicimos a un servidor encargado de enviar y recibir paquetes del protocolo descripto
en la secci\'on \ref{protocolo} a trav\'es del puerto serial.
\\De est\'a forma es cuesti\'on de decidir que implementaci\'on se usa en base a si se quiere
correr en un simulador como Webots, o en el robot f\'isico. Si se quisiese usar otro simulador
u otro tipo de hardware, s\'olo hace falta implementar la interfaz que cumpla con lo establecido e
indicarle a la capa de comportamientos que la utilice para realizar su ejecuci\'on.

\subsubsection{Interfaz con m\'odulo de reconocimiento de objetos usando Visi\'on}
Dado que el desarrollo de los comportamientos lo hicimos separado del desarrollo del
m\'odulo que reconoce objetos utilizando una c\'amara, decidimos hacer una interfaz de
forma tal que se puedan desarrollar de forma paralela.
Como el controlador de comportamientos es \textbf{usuario} del m\'odulo de reconocimiento,
necesita saber en un instante de tiempo $t$, que objetos est\'an siendo reconocidos. Para \'esto
el m\'odulo debe tener una fuente de informaci\'on, en este caso, una c\'amara dado que se usa
Visi\'on.
\\
Desde el punto de vista de la anatom\'ia, \'esto se puede ver como los ojos (c\'amara), la parte del cerebro
encargada de analizar el est\'imulo recibido por los mismos (m\'odulo de reconocimiento) y la parte del cerebro
encargada de analizar los est\'imulos y realizar las acciones (controlador de comportamientos). Si se hubiese
usado alg\'un tipo de conjunto de sensores t\'actiles para reconocer objetos (mano), en vez de visi\'on, 
el m\'odulo de reconocimiento de objetos bien podr\'ia analizar la informaci\'on que ellos proveen e
informar al controlador sobre su an\'alisis.
\\
\'Esta analog\'ia puede llevar a pensar que los sensores de distancia u otros dispositivos que usamos
en este desarrollo bien podr\'ian formar parte de otro m\'odulo, y no se estar\'ia equivocado. La raz\'on
por la cual el m\'odulo de visi\'on est\'a separado y el resto de los sensores no, es que el procesamiento
de una secuencia de im\'agenes es muy complejo y pesado, como se muestra en la secci\'on \ref{vision}.

\newpage
\subsection{Resultados obtenidos}
\label{results}
En \'esta secci\'on se describen los par\'ametros utilizados en la
simulaci\'on para la obtenci\'on de resultados sobre los diferentes
comportamientos. Luego se presentan dichos resultados y se sacan
conclusiones sobre los mismos.
\\
A lo largo del proceso de desarrollo se fueron realizando diferentes simulaciones,
de las cuales observamos defectos en las implementaciones de los comportamientos
as\'i como retardo y detalles que se podr\'ian mejorar en algunos de los mismos
y los fuimos corregiendo. \\
La arena utilizada en dichas simulaciones se puede ver en la figura \ref{fig:arena_sim}.
A lo largo de la misma est\'an ubicadas 15 basuras, un \'area de recarga de bater\'ia
y un \'area de descarga de basura, marcadas por el cilindro naranja. Los valores de
los par\'ametros utilizados se pueden observar en el cuadro \ref{sim_params}. Los
\'angulos est\'an expresados en radianes y las distancias est\'an expresadas en metros,
as\'i como tambi\'en los radios, correcciones y separaciones. La resoluci\'on de la
c\'amara est\'a expresada en pixeles, la de la grilla en unidades y los tiempos est\'an
en milisegundos.
\begin{table}[ht]
	\begin{center}
		\begin{tabular}{|c|c|c|c|c|c|c|c|c|}
			\hline
			Par\'ametro & Descripci\'on del par\'ametro & Valor \\
			\hline
			$P_r(0)$ & Posici\'on Inicial del robot & (-0.295,-0.4) \\
			$O_r(0)$ & Orientaci\'on Inicial del robot & 0 \\
			$R_r$ & Radio del robot & 0.026 \\
			$dbw$ & Distancia entre ruedas & 0.052 \\
			$c_{dbw}$ & Correcci\'on de la distancia entre ruedas & 1.09217109 \\
			$R_l$, $R_r$ & Radios de las ruedas & 0.0205 0.0205 \\
			$c_l$, $c_r$ & Correcciones de los radios de las ruedas & 0.99998702 1.00001297 \\
			$EncRes$ & Resoluci\'on del encoder & 159.23 \\
			$Cd$ & Distancia de la c\'amara al centro del robot & 0.0355 \\
			$Ch$ $C_h$ & Altura de la c\'amara & 0.038 \\
			$FOV_h$ & Field of view horizontal & 1.1 \\
			$ac$ & \'Angulo de la c\'amara & -0.5 \\
			$C_{rw}$, $C_{rh}$ & Resoluci\'on de la c\'amara & 640 x 480 \\
			$A_w$, $A_h$ & Dimensiones del arena & 2 x 1.2 \\
			$A_{rw}$, $A_{rh}$ & Resoluci\'on de la grilla & 100 38\\
			$Fsd$ & Dist. s.de piso del medio al centro del robot & 0.03 \\
			$Fss$ & Separaci\'on entre sensores de piso & 0.01 \\
			$Ts$ & Time Step & 32 \\
			\hline
		\end{tabular}
	\end{center}
	\caption{Par\'ametros utilizados para las simulaciones}
	\label{sim_params}
\end{table}

\begin{figure}[htp]
\begin{center}
\includegraphics[scale=0.25]{comportamientos/arenaSim.png}
\caption{Arena utilizada para las simulaciones}
\label{fig:arena_sim}
\end{center}
\end{figure}

\begin{figure}[htp]
\begin{center}
\includegraphics[scale=0.25]{comportamientos/graphics/areaCoveredVsTimeUse.png}
\caption{\'Area vista y \'Area no vista a lo largo de la simulaci\'on}
\label{fig:seenArea}
\end{center}
\end{figure}

\begin{figure}[htp]
\begin{center}
\includegraphics[scale=0.3]{comportamientos/graphics/BehaviourDistributionPieUse.png}
\caption{Distribuci\'on de tiempos de los comportamientos en la simulaci\'on}
\label{fig:behaviourDistribution}
\end{center}
\end{figure}

\begin{figure}[htp]
\begin{center}
\includegraphics[scale=0.25]{comportamientos/graphics/CollectUse.png}
\caption{Progreso del comportamiento \emph{Recolectar Basura}}
\label{fig:collectEvolution}
\end{center}
\end{figure}

\begin{figure}[htp]
\begin{center}
\includegraphics[scale=0.25]{comportamientos/graphics/AllUse.png}
\caption{Evoluci\'on de los comportamientos a lo largo de la simulaci\'on}
\label{fig:behaviourEvolution}
\end{center}
\end{figure}

\subsubsection{Tiempo promedio en recolectar basura}
%Calcular t min, con un escenario donde las basuras las pongo apenas sale de la base
%Calcular t max, con un escenario donde las basuras estan lo m\'as lejos de la base posible

\subsubsection{Tiempo promedio de wandering}
%(Tiempo en que est\'a activo el wandering)/(Tiempo total de simulaci\'on)

\subsubsection{Distribuci\'on de tiempos entre los diferentes comportamientos}
%(tiempo de comportamiento i)/(Tiempo total de simulaci\'on)

\begin{table}[ht]
	\begin{center}
		\begin{tabular}{|c|c|c|}
			\hline
			Comportamiento & \# total de steps activo & \# veces que se activ\'o \\
			\hline
			Wandering & 64476 & 1330 \\
			Enfocar basura & 2057 & 1384 \\
			Ir a basura & 14832 & 1218 \\
			Recolectar basura & 7166 & 15 \\
			Ir a descargar basura & 76160 & 321 \\
			Ir a recargar bater\'ia & 47188 & 265 \\
			Evitar obst\'aculos & 16843 & 1507 \\
			Salir situaciones indeseadas & 0 & 0 \\
			\hline
		\end{tabular}
	\end{center}
	\caption{Resultados sobre steps que los comportamientos est\'an activos }
	\label{behaviours_stats}
\end{table}

\begin{table}[ht]
	\begin{center}
		\begin{tabular}{|c|c|c|c|}
			\hline
			Comportamiento & Promedio & Desv\'io Est\'andard & \# m\'aximo \\
			\hline
			Wandering & 48.4410 & 225.07466 & 2212 \\
			Enfocar basura & 1.4863 & 3.27591 & 67 \\
			Ir a basura & 12.1773 & 58.44234 & 1222 \\
			Recolectar basura & 477.7333 & 0.70373 & 478 \\
			Ir a descargar basura & 237.2586 & 840.49848 & 6012 \\
			Ir a recargar bater\'ia & 178.0679 & 647.39786 & 3950 \\
			Evitar obst\'aculos & 11.1765 & 54.72354 & 1386 \\
			Salir situaciones indeseadas & 0 & 0 & 0 \\
			\hline
		\end{tabular}
	\end{center}
	\caption{Resultados sobre steps que los comportamientos est\'an continuamente activos}
	\label{behaviours_stats1}
\end{table}

\begin{table}
	\begin{center}
		\begin{tabular}{|c||c|c|c|c|c|c|c|c||c|c|}
		\hline
    	 & W & EB & IAB & RB & DB & R & EO & SSI & Maximo & Ind Max \\
		\hline
		\hline
			W &     0&    303&    176&      1&     12&      5&    833&      0&    833&      7\\
		\hline
			EB &   238&      0&   1036&      2&      1&      0&    107&      0&   1036&      3\\
		\hline
			IAB &    229&    967&      0&     12&      2&      1&      7&      0&    967&      2\\
		\hline
			RB &     12&      3&      0&      0&      0&      0&      0&      0&     12&      1\\
		\hline
			DB &     14&      0&      0&      0&      0&      2&    305&      0&    305&      7\\
		\hline
			R &      9&      0&      0&      0&      1&      0&    255&      0&    255&      7\\
		\hline
			EO &    828&    111&      6&      0&    305&    257&      0&      0&    828&      1\\
		\hline
			SSI &      0&      0&      0&      0&      0&      0&      0&      0&      0&      0\\
		\hline
		\noalign{\smallskip}
		\hline
			Maximo &    828&    967&   1036&     12&    305&    257&    833&      0&      -&      -\\
		\hline
			Ind Max &      7&      3&      2&      3&      7&      7&      1&      0&      -&      -\\
		\hline
		\end{tabular}
	\end{center}
	\caption{Transiciones entre estados}
	\label{transitions}
\end{table}

\subsubsection{Relaci\'on (t\_cargando + t\_ir a base) / tiempo total}
%T busqueda basura/ ( t a base + tiempo carga bateria + t salir base )

\subsubsection{Porcentaje de espacio cubierto}
%esp_cub / (e\sp_cub + esp_nocub)
\begin{table}
	\begin{center}
		\begin{tabular}{|c||c|c|}
		\hline
			$N^\circ$ muestra & (\%) & (\%) Acumulado \\
		\hline
			1 & 0.246 & 0.246 \\
		\hline
			2 & 0.258 & 0.505 \\
		\hline
			3 & 0.081 & 0.586 \\
		\hline
			4 & 0.167 & 0.754 \\
		\hline
			5 & 0.044 & 0.798 \\
		\hline
			6 & 0.011 & 0.809 \\
		\hline
			7 & 0.017 & 0.827 \\
		\hline
			8 & 0.003 & 0.830 \\
		\hline
			9 & 0.038 & 0.869 \\
		\hline
			10 & 0.130 & 1 \\
		\hline
		\end{tabular}
	\end{center}
	\caption{Porcentaje de arena cubierta}
	\label{arena:covered}
\end{table}

\subsubsection{Tiempo cubrir toda la arena}
aaa

\newpage
\subsection{Conclusi\'on}
\label{comp_conclusion}
Conclusi\'on de comportamientos

