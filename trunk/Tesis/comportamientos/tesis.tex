\section{Arquitectura de Comportamientos}

\subsection{Introducci\'on}
Como se dijo en las secciones anteriores, el objetivo de \'este trabajo
es desarrollar un robot aut\'onomo y reactivo que recolecte basura en un entorno
estructurado pero din\'amico, debido a que la arena ( lugar donde se mueve el robot )
es transitado por personas y es al aire libre.

En \'esta secci\'on de \emph{Arquitecturas de Comportamientos} se van
a detallar las acciones que deber\'a hacer el robot y c\'omo organizar las mismas
para cumplir la meta de recolectar basura y ser aut\'onomo, es decir, decidir por
s\'i mismo las acciones a realizar en cada momento y ser capaz de mantenerse cargado
para poder continuar recolectando.

En la secci\'on \ref{inv_prev} se analizar\'an
papers relacionados con este desarrollo, desde la forma de organizar los comportamientos,
la definici\'on y composici\'on de los mismos, hasta su implementaci\'on. En la secci\'on
\ref{arq_prop} se detallar\'a la arquitectura elegida para llevar a cabo y organizar los
comportamientos elegidos, y las ventajas y desventajas de usar la misma. La secci\'on
\ref{comportamientos} detalla uno por uno los comportamientos elegidos que tendr\'a el
robot, y sus correspondientes implementaciones en pseudoc\'odigo. En la secci\'on
\ref{odometry} se explicar\'a que es la odometr\'ia, para qu\'e sirve y qu\'e ventajas y desventajas
tiene usarla. Tambi\'en se detallar\'a el test utilizado para mejorar la eficiencia de la misma.
La secci\'on \ref{interfaces} se describir\'a c\'omo se estructur\'o el controlador del robot
para que se pueda ir desarrollando y probando el mismo con un simulador y minimizar el trabajo
al pasarlo a el robot f\'isico. Los resultados obtenidos de performance y eficiencia del controlador
desarrollado se obtuvieron de la simulaci\'on, y se pueden ver en la secci\'on \ref{results}. Finalmente,
en la secci\'on \ref{comp_conclusion} se sacan conclusiones de los resultados obtenidos y las dificultades
encontradas a lo largo del desarrollo de \'este trabajo pr\'actico final.

\newpage

\subsection{Investigaciones previas}
\label{inv_prev}

\subsubsection{Desarrollo de comportamientos no triviales en robots reales : Robot recolector de basura - Stefano Nolfi}

\subsubsection{Arquitectura de control para un Robot Aut\'onomo M\'ovil - Neves And Oliveira}

\subsubsection{El uso de l\'ogica difusa en la navegaci\'on de un robot aut\'onomo - Alessandro Saffiotti}

\subsubsection{Path Planning usando Algoritmos Gen\'eticos - Salvatore Candido}

\subsubsection{Navegaci\'on predictiva de un robot aut\'onomo - Foka And Trahanias}

\subsubsection{Algoritmo de navegaci\'on y evitamiento de obst\'aculos en un entorno desconocido - Clark Et. al}

\newpage
\subsection{Arquitectura propuesta}
\label{arq_prop}

\newpage

\subsection{Comportamientos e implementaciones}
\label{comportamientos}

Una vez que decidimos utilizar la arquitectura explicada en la secci\'on \ref{arq_prop} para nuestro proyecto, tuvimos que analizar:
\begin{itemize}
\item{}Forma de implementaci\'on de la arquitectura en c\'odigo
\item{}Comportamientos a realizar
\item{}Ordenes de inhibici\'on y supresi\'on entre los mismos
\item{}Forma de implementaci\'on de los mismos
\item{}Orden de implementaci\'on
\end{itemize}

%Detallar que se hizo con cada item
Para implementar la arquitectura decidimos asignarle un $ID$ num\'erico diferente a cada comportamiento.
Tambi\'en tomamos la decisi\'on de elegir como comportamiento activo en el instante $t$,
aquel comportamiento que est\'e activo en ese instante y tenga mayor $ID$,
suprimiendo as\'i el resto de los comportamientos ( con un $ID$ menor ) que podr\'ian estar activos.

%Explicar como se relaciona el requerimiento (robot recolector de basura autonomo) con los comportamientos que pusimos
El requerimiento de este proyecto es la realizaci\'on de un robot aut\'onomo que recolecte
basura de su entorno din\'amico pero estructuralmente fijo. De aqu\'i se infieren algunos de los
comportamientos que debe tener el robot:
\begin{itemize}
	\item{\emph{Recolectar basura}}
	\item{\emph{Recargar bater\'ia}:} Por ser aut\'onomo, debe poder ser capaz de recargarse s\'olo
								para poder continuar con su actividad.
	\item{\emph{Wandering}:} El robot no es controlado por control, ya que es aut\'onomo, por lo que
						debe poder recorrer el entorno por s\'i mismo.
	\item{\emph{Evitamiento de obst\'aculos}:} Debido a la naturaleza del entorno, el robot debe
						ser capaz de navegar sin chocarse contra los l\'imites del entorno
						ni con las personas que circulan por el mismo.
\end{itemize}

El comportamiento de recolectar basura y el requerimiento de la autonom\'ia llevan a su vez
a la aparici\'on de m\'as comportamientos: \emph{Descargar basura} y \emph{Ir hacia basura}.
\\
La figura \ref{fig:architecture} muestra los comportamientos implementados y su orden de jerarqu\'ia.
Se puede ver que hay m\'as comportamientos de los detallados anteriormente. \'Esto se debi\'o a que la forma
en que se implementamos los comportamientos b\'asicos del robot nos requiri\'o el desarrollo
de comportamientos auxiliares.
\\
\begin{figure}[htp]
\begin{center}
\includegraphics[scale=0.5]{comportamientos/behavioursArchitecture.png}
\caption{Arquitectura de Comportamientos}
\label{fig:architecture}
\end{center}
\end{figure}
\\
Primero implementamos \emph{Wandering} debido a, en una primera aproximaci\'on, la sencillez
del mismo. Luego implementamos \emph{Evitamiento de obst\'aculos} para lograr que el robot pueda
navegar sin problemas por el arena. Como el comportamiento de recolectar e ir hacia la basura
dependend\'ia del m\'odulo de reconocimiento de objetos y el mismo estaba siendo desarrollado
en paralelo, se decidi\'o implementar el comportamiento de \emph{Recargar bater\'ia} y \emph{Descargar basura}.
Una vez que tuvimos la primera implementaci\'on funcional del reconomiento de objetos, procedimos a
desarrollar \emph{Ir hacia basura} y \emph{Recolectar Basura}.
\\
A continuaci\'on detallamos los comportamientos indicados en la figura \ref{fig:architecture},
as\'i como la implementaci\'on en \textit{pseudo-codigo} de los mismos y detalles tenidos
en cuenta para la realizaci\'on de los mismos.
%A continuacion vamos a explicar cada comportamiento.....

\subsubsection{Wandering}
\label{wandering}

\paragraph{Detalle del comportamiento}
En una primer aproximaci\'on de \emph{Wandering}, s\'olo nos preocupamos por ir hacia adelante
ya que eventualmente, el robot encuentra un obst\'aculo y realiza un giro cambiando la direcci\'on
del robot.
\\
Los resultados de la simulaci\'on nos indicaron que el robot no recorr\'ia ciertas zonas o las recorr\'ia
despu\'es de un largo tiempo, lo que nos llev\'o a un segundo approach. El mismo tiene en cuenta el hecho
de que el robot posee una c\'amara y por lo tanto se puede llevar un seguimiento de los lugares que m\'as
recientemente visit\'o o las zonas que hace mucho tiempo no visita.
\\
\'Este approach, en cierta forma, genera un modelo del mundo, un hecho que conflict\'ua con uno de
los principios propuestos a seguir en la secci\'on \ref{arq_prop}. Para minimizar el conflicto, decidimos
mantener al m\'inimo la informaci\'on almacenada para el funcionamiento del algoritmo, es decir, por cada
zona de arena s\'olo mantenemos el timestamp de la \'ultima vez que el robot la visit\'o.

\paragraph{Implementaci\'on del comportamiento}
La implementaci\'on del segundo approach, en \emph{pseudo-c\'odigo} es la siguiente:
\begin{verbatim}
por cada paso
    zona = pedir_zona_vista(camara)
    marcar_zona_como_vista(modelodelmundo,zona)
    ultimazonavisitada = pedir_ultima_zona_visitada(modelodelmundo)
    velocidades = calcular_velocidades_de_ruedas(ultimazonavisitada)
    poner_velocidades_en_ruedas(velocidades)
\end{verbatim}


\subsubsection{Enfocar Basura}
\label{focus_garbage}
\paragraph{Detalle del comportamiento}
\paragraph{Implementaci\'on del comportamiento}

\subsubsection{Ir a Basura}
\label{go_to_garbage}
\paragraph{Detalle del comportamiento}
\paragraph{Implementaci\'on del comportamiento}

\subsubsection{Recolectar Basura}
\label{collect_garbage}
\paragraph{Detalle del comportamiento}
\paragraph{Implementaci\'on del comportamiento}

\subsubsection{Ir a zona de descarga de basura}
\label{go_to_unload_zone}

\paragraph{Buscar l\'inea}
\label{find_line}
\subparagraph{Detalle del comportamiento}
\subparagraph{Implementaci\'on del comportamiento}

\paragraph{Entrar a l\'inea}
\label{enter_line}
\subparagraph{Detalle del comportamiento}
\subparagraph{Implementaci\'on del comportamiento}

\paragraph{Seguir l\'inea}
\label{follow_line}
\subparagraph{Detalle del comportamiento}
\subparagraph{Implementaci\'on del comportamiento}

\subsubsection{Descargar Basura}
\label{unload_garbage}
\paragraph{Detalle del comportamiento}
\paragraph{Implementaci\'on del comportamiento}

\subsubsection{Ir a base de recarga de bater\'ia}
\label{go_to_recharge}
\paragraph{Detalle del comportamiento}
\paragraph{Implementaci\'on del comportamiento}

\subsubsection{Cargar Bater\'ia}
\label{recharge_battery}
\paragraph{Detalle del comportamiento}
\paragraph{Implementaci\'on del comportamiento}

\subsubsection{Recalibrarse}
\label{recalibrate}
\paragraph{Detalle del comportamiento}
\paragraph{Implementaci\'on del comportamiento}

\subsubsection{Evitar Obst\'aculos}
\label{avoid_obstacles}
\paragraph{Detalle del comportamiento}
\paragraph{Implementaci\'on del comportamiento}

\subsubsection{Salir de situaciones no deseadas}
\label{out_of_unwanted_situations}
\paragraph{Detalle del comportamiento}
\paragraph{Implementaci\'on del comportamiento}



\newpage
\subsection{Odometr\'ia}
\label{odometry}

\newpage
\subsection{Interfaces con hardware y m\'odulo de reconocimiento de objetos}
\label{interfaces}

\subsubsection{Interfaz con hardware}

\subsubsection{Interfaz con m\'odulo de reconocimiento de objetos}

\newpage
\subsection{Resultados obtenidos}
\label{results}

\subsubsection{Tiempo promedio en recolectar basura}
%Calcular t min, con un escenario donde las basuras las pongo apenas sale de la base
%Calcular t max, con un escenario donde las basuras estan lo m\'as lejos de la base posible

\subsubsection{Tiempo promedio de wandering}
%(Tiempo en que est\'a activo el wandering)/(Tiempo total de simulaci\'on)

\subsubsection{Distribuci\'on de tiempos entre los diferentes comportamientos}
%(tiempo de comportamiento i)/(Tiempo total de simulaci\'on)

\subsubsection{Relaci\'on (t\_cargando + t\_ir a base) / tiempo total}
%T busqueda basura/ ( t a base + tiempo carga bateria + t salir base )

\subsubsection{Porcentaje de espacio cubierto}
%esp_cub / (esp_cub + esp_nocub)

\subsubsection{Tiempo cubrir toda la arena}

\newpage
\subsection{Conclusi\'on}
\label{comp_conclusion}
Conclusi\'on de comportamientos

